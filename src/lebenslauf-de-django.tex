\documentclass[11pt,a4paper,sans]{moderncv}

%% ModernCV themes
\moderncvstyle{classic}
\moderncvcolor{blue}
\renewcommand{\familydefault}{\sfdefault}
\nopagenumbers{}

%% Character encoding
\usepackage[utf8]{inputenc}

%% Adjust the page margins
\usepackage[scale=0.75]{geometry}

%% Personal data
\firstname{Eugen}
\familyname{Ciur}
\title{Sr. Python/Django Developer}
\address{Trusetaler str 90}{12687 Berlin, Deutschland}
\mobile{+49 15736195069}
\email{ciur.eugen@gmail.com}
\homepage{github.com/ciur}


%%------------------------------------------------------------------------------
%% Content
%%------------------------------------------------------------------------------
\begin{document}
    \makecvtitle

    Ich bin Sr. Software-Entwickler mit mehr als 15 Jahren Erfahrung. In den
    letzten 10 Jahren ich beschäftigte mich mit Python basierte Webservices

    \section{Tech Skills}
    \cvdoubleitem{Sprachen}{Python, TypeScript, JavaScript, SQL}{Cloud}{AWS, Docker, Kubernetes, Helmcharts, ArgoCD, Traefik, Terraform, Packer}
    \cvdoubleitem{Open Source Packete}{SQLAlchemy, Redux, React, pytest, pydantic}{Suchmotoren}{SOLR, Elasticsearch}
    \cvdoubleitem{Frameworks}{Django, Flask, FastAPI, DRF}{CVS}{git, mercurial, github, gitlab}
    \cvdoubleitem{Datenbanken}{PostgreSQL, Mysql/MariaDB, SQLite}{Sonstiges}{REST API, OpenAPI, swagger, CI/CD, websockets, HTML, CSS/SASS}
    \cvdoubleitem{Message Queues}{RabbitMQ, Redis, Celery, RQ}{Protocols}{TCP/IP, HTTP, IMAP, POP3, SMTP, LDAP}

    \section{Berufserfahrung}

    \cventry{09.2023 -- 31.10.2024}{Software-Entwickler}{Snom Gmbh}{Berlin}
     { Deutschland }{ Bei Snom habe ich zusammen mit meinem Team eine interne
     Webanwendung zum Testen der Telefon-Firmware entwickelt. Technisch
     gesehen, ist das Backend der Anwendung in Python/Django/RQ. Frontend in
     React. Die Deployment erfolgte mithilfe von Ansible.}

    \cventry{11.2022 -- 09.2023}{Team Leiter}{IONOS Cloud}{Berlin}
     { Deutschland }{ In meiner Rolle als Teamleiter leitete ich ein kleines
     Team von vier Personen das aus zwei Entwicklern, einem QA-Ingenieur und
     mir bestand. Unser Team war verantwortlich für die Wartung
     zwei interne Webdienste (ein in Django und andere in Flask).
     Technisch gesehen war die Anwendung auf Basis
     von Python/Django/Celery (d.h. kein React-Frontend). Die Anwendung wurde im
     internen k8s-Cluster des Unternehmens deployed. Unser Team war für
     die Helmcharts der Anwendung zuständig.}

   \vspace{0.4cm}

    \cventry{05.2021 -- 11.2022}{Sr. Python Entwickler}{IONOS Cloud}{Berlin}
     { Deutschland }{ Ich war verantwortlich für die Entwicklung, Wartung und
     Deployment von zwei internen Anwendungen - eine auf Django basierend,
     eine andere auf Flask + React.}

    \vspace{0.4cm}

    \cventry{12.2019 -- Gegenwart}{Open Source}{max. 4 hours/week}{Berlin}{ Deutschland }
    {Entwicklung eines Open-Source-DMS für digitale Archive}

    \cvitem{Webseite}{\url{https://papermerge.com}}
    \cvitem{Heupt Repo}{\url{https://github.com/ciur/papermerge}}
    \cvitem{Dokumentation}{\url{https://docs.papermerge.io}}
    \cvitem{Github Org}{\url{https://github.com/papermerge}}

  \vspace{0.4cm}

    \cventry{12.2018 -- 12.2020}{Django Lessons}{Youtuber}{Berlin}{ Deutschland }
     { Ich habe 51 Python/Django/Entwicklung/Deployment-Youtube-Videos
     aufgenommen, die relativ beliebt sind. }


    \cvitem{YT}{\url{https://www.youtube.com/@djangolessons4614}}

    \vspace{0.4cm}

    \cventry{12.2012 -- 11.2018}{Software Entwickler}{Oracle}{Berlin}
     { Deutschland }{ Ich war für die Wartung und Entwicklung einer
     Webanwendung zuständig, die als als Front-End für eine größere
     VoIP-Überwachungslösung diente, die die von Hunderten von großen
     Telekommunikationsunternehmen und Dienstleistern genutzt wird. Techstack
      Python für Backend und JavaScript/CSS/HTML für Frontend }

  \vspace{0.4cm}

    \cventry{01.2012 -- 06.2012}{RubyOnRails Entwickler}{Mediapeers}{Berlin}
     { Deutschland }{ In dieser Funktion habe ich eine fullstack Webanwendung
     auf der Grundlage RubyOnRails-Web-Framework. Das Produkt war eine
     Webanwendung zur Verwaltung von Medieninhalten. }

  \vspace{0.4cm}

    \cventry{11.2009 -- 11.2011}{R\&D Engineer}{PrimeTel}{Limassol}{ Zypern}
     { Während meiner Forschungstätigkeit bei PrimeTel leitete ich die
     Entwicklung von IPTV-Commerce-Lösung (in RubyOnRails): IPTV COMMERCE ist
     ein nationales, von der Cyprus Research Stiftung für Forschungsförderung.
     Das Projekt befasste sich mit der Untersuchung der erforderlichen
     Infrastruktur für die Unterstützung der gesamten Dienstleistungen über die
     IPTV-Plattform. }

  \vspace{0.4cm}

    \cventry{12.2007 -- 11.2009}{SME Software Developer}{Amdocs}{Limassol}{ Zypern}{
        Verschiedene Projekte in C/C++/Java/SQL
    }

    \section{Bildung}
    \cventry{2002--2007}{Informatik}{Polytechnische Universität Bukarest}{Bukarest}{Rumänien}{
      An der Politech-Universität in Bukarest habe ich einen Abschluss in Informatik gemacht.
    }

    \section{Languages}

    \cvitemwithcomment{Englisch}{Fließend}{}
    \cvitemwithcomment{Rumänisch }{Muttersprache}{}
    \cvitemwithcomment{German}{B2}{}

\end{document}
