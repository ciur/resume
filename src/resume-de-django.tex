\documentclass[11pt,a4paper,sans]{moderncv}

%% ModernCV themes
\moderncvstyle{classic}
\moderncvcolor{blue}
\renewcommand{\familydefault}{\sfdefault}
\nopagenumbers{}

%% Character encoding
\usepackage[utf8]{inputenc}

%% Adjust the page margins
\usepackage[scale=0.75]{geometry}

%% Personal data
\firstname{Eugen}
\familyname{Ciur}
\title{Sr. Software-Entwickler}
\address{Trusetaler str 90}{12687 Berlin, Germany}
\mobile{+49 15736195069}
\email{ciur.eugen@gmail.com}
\homepage{github.com/ciur}


%%------------------------------------------------------------------------------
%% Content
%%------------------------------------------------------------------------------
\begin{document}
    \makecvtitle

    Ich bin Sr. Software-Entwickler mit mehr als 15 Jahren Erfahrung. In den
    letzten 5 Jahren ich beschäftigte mich mit Webservices, die Python als
    Backend-Sprache verwenden, React als Frontend, und in k8s Cloud-Umgebungen
    eingesetzt werden

    \section{Tech Skills}
    \cvdoubleitem{Sprachen}{Python, Go, TypeScript, C/C++, Java}{Cloud}{AWS, Docker, Kubernetes, Helmcharts, ArgoCD, Traefik, Terraform, Packer}
    \cvdoubleitem{Web Frameworks}{Django, Flask, FastAPI, Django REST Framework, React, Redux, pytest}{CVS}{git, mercurial, github, gitlab}
    \cvdoubleitem{Datenbanken}{PostgreSQL, Mysql/MariaDB, SQLite}{Other}{REST API, OpenAPI, swagger, CI/CD, websockets}
    \cvdoubleitem{Message Queues}{RabbitMQ, Redis}{Protocols}{TCP/IP, HTTP, IMAP, POP3, SMTP}

    \section{Berufserfahrung}

    \cventry{09.2023 -- 31.10.2024}{Software-Entwickler}{Snom Gmbh}{Berlin}
     { Deutschland }{ Bei Snom habe ich zusammen mit meinem Team eine interne
     Webanwendung zum Testen der Telefon-Firmware entwickelt. Das heißt, wir
     haben Backend und Frontend der Anwendung entwickelt und sie in unserem
     internen Proxmox deployed. Technisch gesehen, ist das Backend der
     Anwendung in Python/Django/RQ. Frontend in React. Die Bereitstellung
     erfolgte mithilfe von Ansible (die Docker-Containers auf remote
     VMs starteten).}

    \cventry{11.2022 -- 09.2023}{Team Leiter}{IONOS Cloud}{Berlin}
     { Deutschland }{ In meiner Rolle als Teamleiter leitete ich ein kleines
     Team von vier Personen das aus zwei Entwicklern, einem QA-Ingenieur und
     mir bestand. Unser Team war verantwortlich für die Wartung
     zwei interne Webdienste (ein in Django und andere in Flask).
     Technisch gesehen war die Anwendung auf Basis
     von Python/Django/Celery (d.h. kein React-Frontend). Die Anwendung wurde im
     internen k8s-Cluster des Unternehmens deployed. Unser Team war für
     die Helmcharts der Anwendung zuständig.}

   \vspace{0.4cm}

    \cventry{05.2021 -- 11.2022}{Sr. Python Entwickler}{IONOS Cloud}{Berlin}
     { Deutschland }{ Ich war verantwortlich für die Entwicklung, Wartung und
     Deployment von zwei internen Anwendungen - eine auf Django basierend,
     eine andere auf Flask + React.}

    \vspace{0.4cm}

    \cventry{12.2019 -- present}{Self Employed}{max. 4 hours/week}{Berlin}{ Germany }
    {
      Development and promotion of open source DMS for digital archives. Technically, Papermerge DMS
      is cloud native application - which means it designed to be deployed in k8s environments and
      monitored using prometheus KPI. It uses SOLR as search engine. It can be
      integrated with external authentication systems like Keycloak.
      See "Helm Charts" item below "Helm Charts" for the link with official helm chart
      written (by me) for Papermerge DMS.
    }

    \cvitem{homepage}{\url{https://papermerge.com}}
    \cvitem{main repo}{\url{https://github.com/ciur/papermerge}}
    \cvitem{documentation}{\url{https://docs.papermerge.io}}
    \cvitem{github org}{\url{https://github.com/papermerge}}
    \cvitem{Helm Charts}{\url{https://github.com/papermerge/helm-chart}}

  \vspace{0.4cm}

    \cventry{12.2018 -- 12.2020}{Django Lessons}{Youtuber}{Berlin}{ Germany } {
        I recorded 51 Python/Django/Development/Deployment youtube screencasts which were (and are) relatively popular.
    }


    \cvitem{YT}{\url{https://www.youtube.com/@djangolessons4614}}

    \vspace{0.4cm}

    \cventry{12.2012 -- 11.2018}{Software Engineer}{Oracle}{Berlin}{ Germany }{
        I was in charge of maintenance and development web application which served
        as front-end for a larger VoIP Monitoring solution used
        by hundreds big telcoms and service providers. Mostly I
        dealt with Python based back-end and JavaScript/CSS/HTML based front-
        end. Fullstack web application was written in custom i.e.
        in-house web framework.
    }

  \vspace{0.4cm}

    \cventry{01.2012 -- 06.2012}{RubyOnRails Developer}{Mediapeers}{Berlin}{ Germany }{
        In this role I maintained a full-stack web application based
        on RubyOnRails web framework. Product was web application for management of
        media content.
    }

  \vspace{0.4cm}

    \cventry{11.2009 -- 11.2011}{R\&D Engineer}{PrimeTel}{Limassol}{ Cyprus}{
        During my research activity at PrimeTel I led development of
        IPTV commerce solution (in RubyOnRails): IPTV
        COMMERCE is a nationally funded by the Cyprus Research
        Promotion Foundation project. The project dealt with the study
        of required infrastructure on the support of total of services via
        the IPTV platform.
    }

  \vspace{0.4cm}

    \cventry{12.2007 -- 11.2009}{SME Software Developer}{Amdocs}{Limassol}{ Cyprus}{
        Various projects in C/C++/Java/SQL
    }

    \section{Education}
    \cventry{2002--2007}{Computer Science}{Universitatea Politehnica din București}{Bucharest}{Romania}{
        At Politech University in Bucharest I completed computer science degree
    }

    \cventry{1998--2001}{Informatics}{High School with focus on Math and Informatics}{Chișinău}{Moldova}{}

    \section{Languages}

    \cvitemwithcomment{English}{Fluent}{}
    \cvitemwithcomment{Romanian}{Native}{}
    \cvitemwithcomment{German}{C1}{Can read, write, understand well - but not much experience in speaking}

  \section{Extras}

  \begin{itemize}
    \item 2002 - I won silver medal at 19-th Balkan Mathematical Olympiad, Antalya, Turkey
    \item 2002 - I won 2nd National Prize at Math Olympiad (ro: Olimpiada Republicană de Matematică, Moldova)
    \item 2001 - I won 3rd National Prize at Math Olympiad (ro: Olimpiada Republicană de Matematică, Moldova)
  \end{itemize}

\end{document}
